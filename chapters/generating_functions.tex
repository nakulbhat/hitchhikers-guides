\chapter{Generating Functions}
The general form of a generating function for a sequence \mbox{$\langle a_1, a_2, \cdots a_n\rangle$} is given by
\[
    \sum_{n=0}^{\infty}a_n x^n
\]
Generating functions are used as a way of representing sequences. Dealing directly with sequences is cumbersome, and a generating function makes operations like multiplication trivial. The variable used in a generating function is of no significance, as it is only a formal sum, and we do no worry about convergence as we would with a normal series.

Generating functions can be formed easily and manipulated, but the essence of a generating function lies in the sequence it encodes. For example,
\[
    \langle1,2,3,4\cdots\rangle = \sum_{n=0}^{\infty} nx^n = 0x^0 + 1x^1 + 2x^2 \cdots
\]
is a simple generating function. Suppose we have another generating function
\[
    \langle1,4,9,16 \cdots\rangle =  \sum_{n=0}^{\infty} n^2x^n = 0x^0 + 1x^1 + 4x^2 \cdots
\]
Now, assume we have to find the sequence which combines them using `And' or the multiplication operator. And then we have to find the \mbox{$r^{th}$} term of the new sequence. This would not be easy (but is still possible) to find by just multiplying each of the sequences. But it is made trivial by generating functions.
\begin{align*}
    \langle1,2,3,4\cdots\rangle & \times \langle1,4,9,16 \cdots\rangle  \\
    (0x^0 + 1x^1 + 2x^2 \cdots) & \times (0x^0 + 1x^1 + 4x^2 \cdots)    \\
    (\sum_{n=0}^{\infty} nx^n)  & \times (\sum_{n=0}^{\infty} n^2x^n)\, \\
    \sum_{n=0}^{\infty} (nx^n   & \times n^2 x^n)
    = \sum_{n=0}^{\infty} n^3 x^{2n}
\end{align*}
Now, to find the coefficient of the \mbox{$r^{th}$} term, we substitute \mbox{$r = 2n$},
\[
    a_r = {\left(\frac{r}{2}\right)}^3
\]

\begin{example}
    [Choose one of three]
    Write the generating function for choosing one of three objects \mbox{$a,b,c$}.

    \textbf{Ans:} The generating function can be written as
    \[
        (\underbrace{1}_{\text{not choosing}}+\underbrace{ax}_{\text{choosing}})\underbrace{(1+bx)}_{\text{obj b}}\underbrace{(1+cx)}_{\text{obj c}}
    \]
    Multiplying and taking \mbox{$a=b=c=1$} ways (of representing each obj) we get,
    \[
        1+3x+3x^2+x^3
    \]
    Which upon further evaluation gives
    \[
        {(1+x)}^3
    \]
\end{example}

\section{Combination Generating Functions}

\subsection{Combination without Repetition}

The generating function for choosing \mbox{$r$} objects out of \mbox{$n$} objects is given by
\[
    \Comb{n}{0}x^0 + \Comb{n}{1}x^1 + \Comb{n}{2}x^2 + \cdots
\]
Which basically says that there are \mbox{$\Comb{n}{2}$} ways of choosing two objects out of \mbox{$n$} objects etc.
We can note that this series is equal to the binomial
\[
    {(1+x)}^n
\]

\subsection{Combination with Repetition}
We can represent the possibility of choosing an object infinitely many times using
\[
    1+x+x^2+x^3\cdots = \sum_{i=0}^{\infty} x^i = \frac{1}{1-x}
\]
We do not bother with the convergence of the series as it is only a formal sum.

Suppose we have \mbox{$n$} objects, we can modify the above equation to
\[
    {(1+x+x^2+x^3\cdots)}^n = {(\sum_{i=0}^{\infty} x^i)}^n = {\left(\frac{1}{1-x}\right)}^n
\]
Which simplifies to
\[
    {(1-x)}^{-n}
\]
And when we use the binomial expansion for negative powers,
\[
    \Comb{n+0-1}{0}x^0+\Comb{n+1-1}{1}x^1\cdots = \sum_{i=1}^{\infty}\Comb{n+i-1}{i}x^i
\]
\begin{example}
    [Limited Repetition Choosing]
    Out of 3 objects, the first object can be chosen at most once, the second object can be chosen at most twice and the third object can be chosen at most thrice. Find the number of ways of selecting 4 objects which satisfy the above condition.

    \textbf{Ans:} We get the following generating function based on the above conditions.

    \[
        (1+x) \times (1+x+x^2) \times (1+x+x^2+x^3)
    \]
\end{example}

\begin{example}
    [Repetition, Lower Bound]
    Obtain a generating function to select \mbox{$r$} objects with repetition from five distinct objects with at least two of each type.

    \textbf{Ans:} The generating function is
    \[
        {(x^2+x^3\cdots)}^5 = {(x^2)}^5 {(1+x+x^2\cdots)}^5 = x^{10} {(1-x)}^{-5}
    \]
\end{example}

\begin{table}[h]
    \renewcommand{\arraystretch}{1.5}
    \centering
    \begin{tabularx}{\textwidth}{Xll}
        \toprule
        \textbf{Expanded Form}                                        & \textbf{Summation Form}                                       & \textbf{Generating Function}   \\
        \midrule
        \mbox{$1+x+x^2+\cdots$}                                       & \mbox{$\sum_{n=0}^{\infty} x^n$}                              & \mbox{$\frac{1}{1-x}$}         \\
        \mbox{$1 + \;\Comb{n}{1}x + \;\Comb{n}{2}x^2 + \cdots $}      & \mbox{$\sum_{r=0}^{n} \;\Comb{n}{r} x^r$}                     & \mbox{${(1+x)}^n$}             \\
        \mbox{$1 + \;\Comb{n}{1}x^m + \;\Comb{n}{2}x^{2m} + \cdots $} & \mbox{$\sum_{r=0}^{n} \;\Comb{n}{r} x^{rm}$}                  & \mbox{${(1+x^m)}^{n}$}         \\
        \mbox{$1 - \;\Comb{n}{1}x + \;\Comb{n}{2}x^2 - \cdots $}      & \mbox{$\sum_{r=0}^{n} {(-1)}^r \;\Comb{n}{r} x^r$}            & \mbox{${(1-x)}^n$}             \\
        \mbox{$1 - \;\Comb{n}{1}x + \;\Comb{n+1}{2}x^2 - \cdots$}     & \mbox{$\sum_{r=0}^{\infty} {(-1)}^r \;\Comb{n+r-1}{r} x^r$}   & \mbox{${(1+x)}^{-n}$}          \\
        \mbox{$1 + \;\Comb{n}{1}x + \;\Comb{n+1}{2}x^2 + \cdots$}     & \mbox{$\sum_{r=0}^{\infty} \;\Comb{n+r-1}{r} x^r$}            & \mbox{${(1-x)}^{-n}$}          \\
        \mbox{$1+x+x^2+\cdots+x^m$}                                   & \mbox{$\sum_{n=0}^{m} x^n$}                                   & \mbox{$\frac{1-x^{m+1}}{1-x}$} \\
        \mbox{$1+nx+n^2\frac{x^2}{2!}+n^3\frac{x^3}{3!}+\cdots$}      & \mbox{$\sum_{r=0}^{\infty} n^r \frac{x^r}{r!}$}               & \mbox{$e^{nx}$}                \\
        \mbox{$1+x+nx+\frac{n(n-1)}{2!}x^2+\cdots$}                   & \mbox{$\sum_{n=0}^{\infty} \Perm{n}{r} \frac{x^r}{r!}$}       & \mbox{${(1+x)}^r$}             \\
        \mbox{$1+\frac{x^2}{2!}+\frac{x^4}{4!}+\cdots$}               & \mbox{$\sum_{n=0}^{\infty} \frac{x^{2n}}{(2n)!}=\cos(x)$}     & \mbox{$\frac{e^x+e^{-x}}{2}$}  \\
        \mbox{$x+\frac{x^3}{3!}+\frac{x^5}{5!}+\cdots$}               & \mbox{$\sum_{n=0}^{\infty} \frac{x^{2n+1}}{(2n+1)!}=\sin(x)$} & \mbox{$\frac{e^x+e^{-x}}{2}$}  \\
        \bottomrule
    \end{tabularx}
    \caption{Basic Generating Functions}\label{tab:basic_generating_functions}
\end{table}


\begin{example}
    [Marbles from a Pile]
    Using generating functions, find the number of ways to select 10 marbles from a large pile of blue, red and white marbles if
    \begin{enumerate}
        \item Selection has at least two marbles of each colour,
        \item Solution has at most two red marbles,
        \item The selection has an even number of blue marbles.
    \end{enumerate}

    \textbf{Ans:} \begin{enumerate}
        \item \begin{align*}
                   & {(x^2+x^3\cdots)}^3                \\
                   & x^6 {(1+x+x^2\cdots)}^3            \\
                   & x^6 {\left(\frac{1}{1-x}\right)}^3 \\
                   & x^6 {(1-x)}^{-3}
              \end{align*}
              using the general term for \mbox{${(1-x)}^{-3}$}, we get
              \[
                  x^6\quad \Comb{3+r-1}{r} {(-1)}^r x^r
              \]
              Now, to get the term for \mbox{$x^{10}$}, put \mbox{$r=4$}
              \[
                  x^6\quad\Comb{6}{4} x^4
              \]
              where we get the coefficient, \mbox{$\Comb{6}{4}$} ways.

        \item \begin{align*}
                   & (1+x+x^2){(1+x+x^2\cdots)}^2             \\
                   & (1+x+x^2){(1-x)}^{-2}                    \\
                   & (1+x+x^2)\left(\Comb{2+r-1}{r}x^r\right)
              \end{align*}
              multiplying with each term on the left, we get three terms where \mbox{$x^{10}$} is possible
              \[
                  \Comb{2+10-1}{10}x^{10} + \Comb{2+9-1}{9}x^{10} + \Comb{2+8-1}{8}x^{10}
              \]
              now we get \mbox{$\Comb{11}{10} + \Comb{10}{9}+ \Comb{9}{8} = 30$} ways

        \item \begin{align*}
                   & (1+x^2+x^4+x^6+x^8+x^{10}){(1+x+x^2\cdots)}^2\, \\
                   & (1+x^2+x^4+x^6+x^8+x^{10}){(1-x)}^{-2}          \\
                   & (1+x^2+x^4+x^6+x^8+x^{10})\Comb{2+r-1}{r}x^r\,  \\
              \end{align*}
              Now, finding coefficients of \mbox{$x^{10}$},
              \[
                  \Comb{11}{10} + \Comb{9}{8} + \Comb{7}{6} +\Comb{5}{4} + \Comb{3}{2} + \Comb{1}{0} = 36 \text{ ways}
              \]
    \end{enumerate}
\end{example}

\begin{example}
    [Examiners' Problem]
    In how many ways can an examiner distribute 30 marks among 8 questions such that all questions get at least 2 marks?

    \textbf{Ans:}
    \begin{gather*}
        {(x^2 + x^3 \cdots)}^8 \\
        x^{16}\quad {(1 - x)}^{-8}\\
        x^{16}\quad\Comb{8 + r - 1}{r} x^r  \tag*{r = 14,} \\
        \Comb{21}{14} x^{30}
    \end{gather*}
\end{example}
\begin{example}
    [Ice Cream Selection]
    How many ways are there to select 12 ice creams from 5 types of sundaes with at most 4 of each type?
    \textbf{Ans:}
    \begin{gather*}
        {(1 + x + x^2 + x^3 + x^4 )}^5 \\
        {\left(\frac{1 - x^5}{1 - x}\right)}^5 \\
        {(1 - x^5)}^5 {(1 - x)}^{-5}\\
        {( - 1)}^r\: \Comb{5}{r} x^5r \times \Comb{5 + r - 1}{r} x^r \\
        {( - 1)}^r\: \Comb{5}{r} \Comb{5 + r - 1}{r}\: x^{6r} \tag*{r = 2,} \\
        \Comb{5}{2} \Comb{6}{2} x^{12}
    \end{gather*}
\end{example}

\begin{example}
    [Collecting Money from People]
    Find the number of ways to collect \mathdollar15 from 20 different people if 19 people can give \mathdollar1 or nothing and the last person can give either \mathdollar1 or \mathdollar5 or nothing

    \textbf{Ans:}\begin{gather*}
        {(1 + x)}^{19} {(1 + x + x^5)}\\
        \Comb{19}{r} x^r {(1 + x + x^5)} \tag*{r = 15,14,10}\\
        {(\Comb{19}{15} + \Comb{19}{14} +\Comb{19}{10})} x^{20}
    \end{gather*}
\end{example}


\begin{example}
    [Distributing Oranges]
    While Shopping on saturday, Mary bought 12 oranges for her children, Grace, George and Frank. In how many ways can she distribute the oranges so that Grace gets at least 4, George and Frank get at least 2 and Frank gets no more than 5.

    \textbf{Ans:}
    \begin{gather*}
        {(x^4 + x^5 \cdots)}{(x^2 + x^3 \cdots )}{(x^2 + x^3 + x^4 + x^5)} \\
        x^4 {(1 - x)}^{-1} \quad x^2 {(1 - x)}^{-1} \quad x^2 \frac{1 - x^4}{1 - x}\\
        x^8 {(1 - x)}^{-3} (1 - x^4) \\
        x^8 \quad \Comb{3 + r - 1}{r} x^r (1 - x^4) \tag*{r = 4,0}\\
        \Comb{6}{4} - 1
    \end{gather*}
\end{example}

\begin{example}
    [Partial Restriction]
    Find the number of ways to distribute 25 identical balls into 7 distinct boxex if the first box can have not more than 10 balls, but the other boxes have no restriction.

    \textbf{Ans:}
    \begin{gather*}
        {(1 + x + x^2\cdots + x^{10})} {(1 + x + x^2\cdots )}^6\\
        {\left(\frac{1 - x^{11}}{1 - x}\right)} {(1 - x)}^{-6} \\
        {(1 - x^{11})} {(1 - x)}^{-7} \\
        {(1 - x^{11})} \Comb{6 + r - 1}{r} x^r \tag*{r = 25,14} \\
        \left(\Comb{30}{25} - \Comb{19}{14} \right) x^{25}
    \end{gather*}
\end{example}

\section{Permutation Genetating Functions}
\subsection{Permutation Without Repetition}

We know that
\begin{equation*}
    \Perm{n}{r} = \frac{\Comb{n}{r}}{r!}
\end{equation*}
And we can see in the binomial distribution,
\begin{equation*}
    \sum_{r = 0}^{n} \Comb{n}{r} x^r = \sum_{r = 0}^{n} \Perm{n}{r} \frac{x^r}{r!}
\end{equation*}
So, to find the possible permutations, we take the coefficient of \mbox{\(\frac{x^r}{r!}\)} instead of \mbox{\(x^r\)}.

\subsection{Permutation With Repetition}
The sequence encoding permutations with repetion is given by
\begin{equation*}
    \langle 1,n,n^2,n^3 \cdots\rangle
\end{equation*}
And the series encoding this sequence,
\begin{gather*}
    1 + nx + n^2\frac{x^2}{2!} + n^3 \frac{x^3}{3!} \cdots \\
    \sum_{r = 0}^{\infty} n^r \frac{x^r}{r!} = e^{nx}
\end{gather*}

\begin{example}
    [ENGINE Permutation]
    Find the number of arrangements of 4 letters from the word ENGINE

    \textbf{Ans:} We can note that the letters N and E are repeating twice, while G and I are repeating once each. So,
    \begin{gather*}
        \underbrace{{(1 + x)}^2}_{\text{G and I}} \underbrace{{\left(1 + x + \frac{x^2}{2!}\right)}^2}_{\text{N and E}}\\
        (1 + 2x + x^2) \cdot \left(1 + 2x + \frac{3x^2}{2} + x^3 + \frac{x^4}{4}\right) \tag*{finding \mbox{\(x^4\)}} \\
        24 \times \left(1 +\frac{1}{4} + 1 + 2\right) \frac{x^4}{4!}\\
        102 \text{ ways}
    \end{gather*}
\end{example}

\begin{example}
    [Distributing Toys]
    How many ways are there to distribute 8 toys among 4 children if the first child should get at least 2?

    \textbf{Ans:} \begin{gather*}
        \left(\frac{x^2}{2!} + \frac{x^3}{3!}\cdots\right) {\left(1 + x + \frac{x^2}{2!}\cdots\right)}^3\\
        (e^x - 1 - x) e^{3x} \marginnote{A little bit of complicated algebra has happened here, but when you do it on your own it will be clearer.}\\
        e^{4x} - e^{3x} - xe^{3x} \\
        ( 4^r - 3^r - r3^{r - 1} )\frac{x^r}{r!} \tag*{r = 8} \\
        4^8 - 3^8 - 8 \times 3^7
    \end{gather*}
\end{example}

\begin{example}
    [Ternery Sequence]
    How many 4 digit ternery sequences are there with at least one 0, one 1, one 2.

    \textbf{Ans:} A ternery sequence is a sequence comprising of a three letter alphabet.
    \begin{gather*}
        {\left(x + \frac{x^2}{2!} +\frac{x^3}{3!} \cdots \right)}^3\\
        {(e^x - 1)}^3 \\
        e^{3x} + 3e^{2x} + 3e^x - 1 \\
        3^r \frac{x^r}{r!} - 3 \times 2^r \frac{x^r}{r!} + 3 \times \frac{x^r}{r!} - 1 \tag*{r = 4}\\
        3^4 - 3 \times 2^4 + 3 \text{ ways}
    \end{gather*}
\end{example}

\begin{example}
    [Ship Singal Permutation]
    A ship carries 48 flags, 12 each of the colour white, red, black and blue. 12 of these flags are placed on a vertical pole, in order to communicate a signals to other ships. How many of these signals use\begin{enumerate}
        \item an even number of blue flags and an odd number of black flags.
        \item at least three white flags or no white flags at all.
    \end{enumerate}
    \textbf{Ans:}\begin{enumerate}
        \item \begin{gather*}
                  \left(1 + \frac{x^2}{2!} + \frac{x^4}{4!} \cdots \right) \left(x + \frac{x^3}{3!} + \frac{x^5}{5!}\cdots\right) {\left(1 + x +\frac{x^2}{2!} \cdots\right)}^2\\
                  (\cos x) (\sin x) {(e^x)}^2 \\
                  \frac{e^x + e^{ - x}}{2} \times  \frac{e^x - e^{-x}}{2} \times  e^2x\\
                  \frac{1}{4} (e^{2x} - e^{ - 2x}) e^{2x}\\
                  \frac{1}{4} (e^{4x} - 1) \tag*{r = 12}\\
                  \frac{1}{4} 4^{12} \frac{x^{12}}{12!} = 4^{11} \text{ ways}
              \end{gather*}

        \item \begin{gather*}
                  \left( 1 + \frac{x^{3}}{3!} \cdots \right) {\left( 1 + x + \frac{x^{2}}{2!} \cdots \right)}^3\\
                  \left( e^x - x - \frac{x^2}{2!}\right) e^{3x}\\
                  e^{4x} - xe^{3x} - \frac{x^2}{2!} e^{3x} \\
                  4^r \frac{x^r}{r!} - 3^r \frac{x^{r+1}}{r!} - 3^r \frac{x^{r+2}}{r!} \marginnote{Note that we have to find the coeff. of \mbox{\(\frac{x^r}{r!}\)}, not just \mbox{\(x^r\)}.} \tag*{r=12,11,10} \\
                  \left( 4^{12} - 3^{12} \cdot 12 - 3^{12} \cdot 12 \cdot 11  \right) \frac{x^{12}}{12!} 
              \end{gather*}
    \end{enumerate}
\end{example}
