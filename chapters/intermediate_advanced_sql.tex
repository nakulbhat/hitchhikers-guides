\chapter{Intermediate and Advanced SQL}

\section{Joins}

Joins allow combining data from multiple tables.

\begin{minted}{sql}
SELECT students.id, students.name, courses.course_name 
FROM students 
JOIN enrollments ON students.id = enrollments.student_id 
JOIN courses ON enrollments.course_id = courses.id;
\end{minted}

\begin{exampletcb}{Using Joins}{ex:joins}
Retrieves student names and their enrolled courses by joining tables.
\end{exampletcb}

\section{Subqueries}

Subqueries allow nesting one query inside another.

\begin{minted}{sql}
SELECT name FROM students WHERE id IN (SELECT student_id FROM enrollments);
\end{minted}

\begin{exampletcb}{Using Subqueries}{ex:subqueries}
Retrieves names of students who are enrolled in at least one course.
\end{exampletcb}

\section{Views}

Views are virtual tables based on SQL queries.

\begin{minted}{sql}
CREATE VIEW student_courses AS 
SELECT students.name, courses.course_name 
FROM students 
JOIN enrollments ON students.id = enrollments.student_id 
JOIN courses ON enrollments.course_id = courses.id;
\end{minted}

\begin{exampletcb}{Creating a View}{ex:view}
Creates a view `student\_courses' that stores student names and their enrolled courses.
\end{exampletcb}

\section{Stored Procedures}

Stored procedures allow defining reusable SQL code blocks.

\begin{minted}{sql}
DELIMITER //
CREATE PROCEDURE GetStudentCourses(IN student_id INT)
BEGIN
    SELECT courses.course_name 
    FROM enrollments 
    JOIN courses ON enrollments.course_id = courses.id 
    WHERE enrollments.student_id = student_id;
END //
DELIMITER ;
\end{minted}

\begin{exampletcb}{Creating a Stored Procedure}{ex:stored-procedure}
Defines a procedure `GetStudentCourses' that retrieves all courses of a student.
\end{exampletcb}
