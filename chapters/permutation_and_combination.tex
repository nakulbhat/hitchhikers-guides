\chapter{Combinatorics}

\begin{abstract}
    Combinatorics is a field of mathematics that deals with the counting of discrete objects. Any applications of mathematics that deals with selection or arrangements of objects comes under the purview of combinatorics. \marginnote{Whether you use permutation or combination is irrelevant in this case.}Combinatorics is essential for calculating probability, as it is often necessary to know the total number of possibilities to calculate probability.
\end{abstract}

\begin{theorem}[The Fundamental Theorem Of Counting]
    If there are two events that happen  in succession, and there are \mbox{$m$} ways to do the first one; and \mbox{$n$} ways to do the second one, then there are \mbox{$m\times n$} ways of completing the tasks in succession
\end{theorem}

\begin{example}
    [Throwing 2 Dice]
    Suppose we have two dice which are thrown successively, and the readings are noted. there are 6 possibilities in each die, and 36 possibilities in total. Each of those possibilities are represented in the following matrix.
    \begin{equation*}
        \begin{pmatrix}
            (1,1)  & (1,2)  & \dots  & (1,6)  \\
            (2,1)  & (2,2)  & \dots  & (2,6)  \\
            \vdots & \vdots & \ddots & \vdots \\
            (6,1)  & (6,2)  & \dots  & (6,6)
        \end{pmatrix}
    \end{equation*}
\end{example}

\section{Permutation}
Permutation is the study of arrangement of objects. Formally, it is defined as the act of arranging members of a given set into a sequence or order.

\begin{intuition}
[Factorials in Combinatorics]
Factorials are omnipresent in combinatorics. This is because factorials allow us to elegantly represent a particular scenario that occurs often in problems of combinatorics. Suppose you are given the responsibility of arranging 5 people into 5 seats. Logically, you have 5 choices for seat 1, 4 for seat 2 and so on. Thus you have stumbled upon \mbox{$5!$}.
\begin{equation*}
    \boxed{5} \times \boxed{4} \times \boxed{3} \times \boxed{2} \times \boxed{1}  = 5! = 120 \text{ ways}
\end{equation*} 
\end{intuition}